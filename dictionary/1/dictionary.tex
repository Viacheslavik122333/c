\documentclass[a4paper, 10pt]{article}

\usepackage{import}

% Корректность отображения всех шрифтов, кодировок и мат. символов
\usepackage[T2A]{fontenc}
\usepackage[utf8]{inputenc}
\usepackage[english, russian]{babel}
\usepackage{amssymb, amsmath, amsthm, mathtools}

% Отображение содержания
\usepackage{tocloft}

% Вставка картинок
\usepackage{graphicx}
\usepackage{tikz}
\usepackage{tkz-euclide}
\usepackage{asymptote}


\usepackage{wrapfig}        % Огибание картинок текстом
\usepackage{cancel}         % Зачёркивания
\usepackage{indentfirst}    % Отступ у первого абзаца
\usepackage{xcolor}         % Цвета
\setlength{\parskip}{.5ex}  % Отступы между абзацами
\usepackage{enumitem}       % Работа со списками
% \usepackage{minted}       % Вставка блоков кода

\usepackage{hyperref}       % гиперссылки
\definecolor{linkcolor}{HTML}{225ae2} % Цвет ссылок
\definecolor{urlcolor}{HTML}{225ae2} % Цвет гиперссылок
\hypersetup{
    pdfstartview=FitH, 
    linkcolor=linkcolor,
    urlcolor=urlcolor,
    colorlinks=true}
\setlength{\arrayrulewidth}{0.5mm} %Толщина линейки в таблицах
\setlength{\tabcolsep}{18pt} %Разделение между столбцами в таблице

% Отступы на странице
\usepackage[left=2cm, right=1.5cm, top=2cm, bottom=2cm]{geometry}

\usepackage{cmap}            % Русский поиск в PDF документе
\usepackage{etoolbox}
\usepackage{soul}            % Разряженный текст \so{} и подчеркивание \ul{}
\usepackage{soulutf8}        % Поддержка UTF8 в soul

\usepackage{titlesec}        % Форматирование заголовков
\titleformat{\section}{\Large \bfseries}{\thesection}{1em}{}
\titleformat{\subsection}{\Large\bfseries}{\thesubsection}{1em}{}
\titleformat{\subsubsection}{\large\bfseries}{\thesubsubsection}{1em}{}

\newcommand{\R}{\mathbb R}
\newcommand{\Q}{\mathbb Q}
\newcommand{\Z}{\mathbb Z}
\newcommand{\N}{\mathbb N}
\newcommand{\CC}{\mathbb C}
\newcommand{\F}{\mathbb F}
\newcommand{\aug}{\fboxsep=-\fboxrule\!\!\!\fbox{\strut}\!\!\!}
\newcommand{\sgn}{\operatorname{sgn}}
\newcommand{\id}{\mathrm{id}}
\renewcommand{\phi}{\varphi}
\renewcommand{\epsilon}{\varepsilon}

\newsavebox{\boxedalignbox}
\newenvironment{boxedalign*}
  {\begin{equation*}\begin{lrbox}{\boxedalignbox}$\begin{aligned}}
  {\end{aligned}$\end{lrbox}\fbox{\usebox{\boxedalignbox}}\end{equation*}}

\newcommand\tab[1][.5cm]{\hspace*{#1}}

% Подписи для матриц
\newcommand\undermat[2]{\makebox[0pt][l]{$\smash{\underbrace
{\phantom{\begin{matrix}#2\end{matrix}}}_{\text{$#1$}}}$}#2}
\newcommand\overmat[2]{\makebox[0pt][l]{$\smash{\overbrace
{\phantom{\begin{matrix}#2\end{matrix}}}^{\text{$#1$}}}$}#2}

% Значек "пусть"
\newlength{\tempheight}  
\newcommand{\Let}[0]{  
\mathbin{\text{\settoheight{\tempheight}{\mathstrut}\raisebox{0.5\pgflinewidth}{%
\tikz[baseline,line cap=round,line join=round] \draw (0,0) --++ (0.4em,0) --++ (0,1.5ex) --++ (-0.4em,0);
}}}}


% \newcounter{lemcount}
% \newcounter{thcount}
% \newcounter{offercount}
% \newcounter{concount}
% \newcounter{subthcount}
% \newcounter{defcount}

\theoremstyle{definition}
\newtheorem*{definition}{Определение}
% \newtheorem{definitionnum}[defcount]{Определение}
\newtheorem*{example}{Примеры}
\newtheorem*{example1}{Пример}
\newtheorem*{exercise}{Упражнение}


\theoremstyle{plain}
\newtheorem*{theorem}{Теорема}
% \newtheorem{theoremnum}[thcount]{Теорема}
\newtheorem*{consequense}{Следствие}
\newtheorem*{consequenses}{Следствия}
% \newtheorem{consequensenum}[concount]{Следствие}
\newtheorem*{lemma}{Лемма}
% \newtheorem{lemmanum}[lemcount]{Лемма}
\newtheorem*{subtheorem}{Утверждение}
% \newtheorem{subtheoremnum}[subthcount]{Утверждение}
\newtheorem*{algorithm}{Алгоритм}
\newtheorem*{properties}{Свойства}
\newtheorem*{properties1}{Свойство}


\theoremstyle{remark}
\newtheorem*{remark}{Замечание}
\newtheorem*{offer}{Предложение}
% \newtheorem{offernum}[offercount]{Предложение}

\begin{document}
\begin{center}
    \textbf{{\LARGE Словарь математических терминов} }    
\end{center}

\section{Section}

\subsection{set – множество}

\begin{itemize}
    \item \textbf{ADJ.}
    \begin{itemize}
        \item finite set – конечное множество
        \item infinite set – бесконечное множество
        \item empty set – пустое множество
        \item subset – подмножество
        \item proper subset – собственное подмножество
    \end{itemize}
    
    \item \textbf{SET + NOUN}
    \begin{itemize}
        \item set notation – обозначение множества
        \item set equality – равенство множеств
        \item set operations – операции над множествами
        \item set membership – принадлежность множеству
    \end{itemize}
    
    \item \textbf{SET + OF + NOUN}
    \begin{itemize}
        \item set of solutions – множество решений
        \item set of points – множество точек
        \item set of natural numbers – множество натуральных чисел
        \item set of real numbers – множество действительных чисел
    \end{itemize}
    
    \item \textbf{NOUN + SET}
    \begin{itemize}
        \item solution set – множество решений
        \item plane set – множество на плоскости
        \item power set – булеан (множество всех подмножеств)
    \end{itemize}
    
    \item \textbf{NOUN + OF + SET}
    \begin{itemize}
        \item element of a set – элемент множества
        \item subset of a set – подмножество множества
        \item union of sets – объединение множеств
        \item intersection of sets – пересечение множеств
    \end{itemize}
    
    \item \textbf{PHRASES}
    \begin{itemize}
        \item the set of all natural numbers – множество всех натуральных чисел
        \item the set of real numbers – множество действительных чисел
        \item the set of integers – множество целых чисел
        \item the set of rational numbers – множество рациональных чисел
    \end{itemize}
\end{itemize}

\subsection{union – объединение}

\begin{itemize}
    \item \textbf{ADJ.}
    \begin{itemize}
        \item disjoint union – непересекающееся объединение
    \end{itemize}
    
    \item \textbf{UNION + NOUN}
    \begin{itemize}
        \item union of sets – объединение множеств
    \end{itemize}
    
    \item \textbf{PHRASES}
    \begin{itemize}
        \item the union of A and B – объединение A и B
        \item the union of intervals – объединение интервалов
    \end{itemize}
\end{itemize}

\subsection{intersection – пересечение}

\begin{itemize}
    \item \textbf{ADJ.}
    \begin{itemize}
        \item empty intersection – пустое пересечение
    \end{itemize}
    
    \item \textbf{INTERSECTION + NOUN}
    \begin{itemize}
        \item intersection of sets – пересечение множеств
    \end{itemize}
    
    \item \textbf{PHRASES}
    \begin{itemize}
        \item the intersection of A and B – пересечение A и B
        \item the intersection of intervals – пересечение интервалов
    \end{itemize}
\end{itemize}

\subsection{difference – разность}

\begin{itemize}
    \item \textbf{ADJ.}
    \begin{itemize}
        \item symmetric difference – симметрическая разность
    \end{itemize}
    
    \item \textbf{DIFFERENCE + NOUN}
    \begin{itemize}
        \item difference of sets – разность множеств
    \end{itemize}
    
    \item \textbf{PHRASES}
    \begin{itemize}
        \item the difference between A and B – разность между A и B
        \item the difference of intervals – разность интервалов
    \end{itemize}
\end{itemize}

\subsection{element – элемент}

\begin{itemize}
    \item \textbf{ADJ.}
    \begin{itemize}
        \item arbitrary element – произвольный элемент
        \item specific element – конкретный элемент
    \end{itemize}
    
    \item \textbf{ELEMENT + NOUN}
    \begin{itemize}
        \item element of a set – элемент множества
    \end{itemize}
    
    \item \textbf{PHRASES}
    \begin{itemize}
        \item an element of the set – элемент множества
        \item the elements of a list – элементы списка
    \end{itemize}
\end{itemize}

\subsection{subset – подмножество}

\begin{itemize}
    \item \textbf{ADJ.}
    \begin{itemize}
        \item proper subset – собственное подмножество
    \end{itemize}
    
    \item \textbf{SUBSET + NOUN}
    \begin{itemize}
        \item subset relation – отношение подмножества
    \end{itemize}
    
    \item \textbf{PHRASES}
    \begin{itemize}
        \item a subset of the set – подмножество множества
        \item the subset of natural numbers – подмножество натуральных чисел
    \end{itemize}
\end{itemize}

\subsection{plane set – множество на плоскости}

\begin{itemize}
    \item \textbf{ADJ.}
    \begin{itemize}
        \item finite plane set – конечное множество на плоскости
        \item infinite plane set – бесконечное множество на плоскости
    \end{itemize}
    
    \item \textbf{PLANE SET + NOUN}
    \begin{itemize}
        \item plane set notation – обозначение множества на плоскости
    \end{itemize}
    
    \item \textbf{PHRASES}
    \begin{itemize}
        \item the plane set of points – множество точек на плоскости
        \item the plane set of solutions – множество решений на плоскости
    \end{itemize}
\end{itemize}

\subsection{interval – интервал}

\begin{itemize}
    \item \textbf{ADJ.}
    \begin{itemize}
        \item open interval – открытый интервал
        \item closed interval – замкнутый интервал
        \item half-open interval – полуоткрытый интервал
    \end{itemize}
    
    \item \textbf{INTERVAL + NOUN}
    \begin{itemize}
        \item interval notation – обозначение интервала
    \end{itemize}
    
    \item \textbf{PHRASES}
    \begin{itemize}
        \item the interval from a to b – интервал от a до b
        \item the interval of real numbers – интервал действительных чисел
    \end{itemize}
\end{itemize}

\subsection{function – функция}

\begin{itemize}
    \item \textbf{ADJ.}
    \begin{itemize}
        \item real function – действительная функция
        \item continuous function – непрерывная функция
        \item differentiable function – дифференцируемая функция
    \end{itemize}
    
    \item \textbf{FUNCTION + NOUN}
    \begin{itemize}
        \item function domain – область определения функции
        \item function graph – график функции
    \end{itemize}
    
    \item \textbf{PHRASES}
    \begin{itemize}
        \item the function of x – функция от x
        \item the graph of the function – график функции
    \end{itemize}
\end{itemize}

\subsection{graph – график}

\begin{itemize}
    \item \textbf{ADJ.}
    \begin{itemize}
        \item function graph – график функции
        \item linear graph – линейный график
    \end{itemize}
    
    \item \textbf{GRAPH + NOUN}
    \begin{itemize}
        \item graph of a function – график функции
    \end{itemize}
    
    \item \textbf{PHRASES}
    \begin{itemize}
        \item the graph of the equation – график уравнения
        \item the graph of the line – график прямой
    \end{itemize}
\end{itemize}

\subsection{equation – уравнение}

\begin{itemize}
    \item \textbf{ADJ.}
    \begin{itemize}
        \item quadratic equation – квадратное уравнение
        \item linear equation – линейное уравнение
    \end{itemize}
    
    \item \textbf{EQUATION + NOUN}
    \begin{itemize}
        \item equation solution – решение уравнения
    \end{itemize}
    
    \item \textbf{PHRASES}
    \begin{itemize}
        \item the solution of the equation – решение уравнения
        \item the equation of the line – уравнение прямой
    \end{itemize}
\end{itemize}

\subsection{variable – переменная}

\begin{itemize}
    \item \textbf{ADJ.}
    \begin{itemize}
        \item real variable – действительная переменная
        \item independent variable – независимая переменная
        \item dependent variable – зависимая переменная
    \end{itemize}
    
    \item \textbf{VARIABLE + NOUN}
    \begin{itemize}
        \item variable value – значение переменной
    \end{itemize}
    
    \item \textbf{PHRASES}
    \begin{itemize}
        \item the variable x – переменная x
        \item the value of the variable – значение переменной
    \end{itemize}
\end{itemize}

\subsection{domain – область определения}

\begin{itemize}
    \item \textbf{ADJ.}
    \begin{itemize}
        \item function domain – область определения функции
    \end{itemize}
    
    \item \textbf{DOMAIN + NOUN}
    \begin{itemize}
        \item domain of definition – область определения
    \end{itemize}
    
    \item \textbf{PHRASES}
    \begin{itemize}
        \item the domain of the function – область определения функции
        \item the domain of real numbers – область действительных чисел
    \end{itemize}
\end{itemize}

\subsection{range – область значений}

\begin{itemize}
    \item \textbf{ADJ.}
    \begin{itemize}
        \item function range – область значений функции
    \end{itemize}
    
    \item \textbf{RANGE + NOUN}
    \begin{itemize}
        \item range of values – диапазон значений
    \end{itemize}
    
    \item \textbf{PHRASES}
    \begin{itemize}
        \item the range of the function – область значений функции
        \item the range of real numbers – область действительных чисел
    \end{itemize}
\end{itemize}

\section{Section}

\subsection{function – функция}

\begin{itemize}
    \item \textbf{ADJ.}
    \begin{itemize}
        \item real function – действительная функция
        \item one-one function – инъективная функция
        \item onto function – сюръективная функция
        \item inverse function – обратная функция
        \item composite function – композиция функций
        \item identity function – тождественная функция
        \item distance function – функция расстояния
        \item parametrisation function – функция параметризации
        \item linear function – линейная функция
        \item quadratic function – квадратичная функция
    \end{itemize}
    
    \item \textbf{FUNCTION + NOUN}
    \begin{itemize}
        \item function domain – область определения функции
        \item function codomain – область значений функции
        \item function image – образ функции
        \item function rule – правило функции
        \item function graph – график функции
        \item function composition – композиция функций
    \end{itemize}
    
    \item \textbf{FUNCTION + OF + NOUN}
    \begin{itemize}
        \item function of a variable – функция переменной
        \item function of two variables – функция двух переменных
        \item function of time – функция времени
    \end{itemize}
    
    \item \textbf{NOUN + FUNCTION}
    \begin{itemize}
        \item identity function – тождественная функция
        \item distance function – функция расстояния
        \item parametrisation function – функция параметризации
        \item transformation function – функция преобразования
    \end{itemize}
    
    \item \textbf{NOUN + OF + FUNCTION}
    \begin{itemize}
        \item image of a function – образ функции
        \item inverse of a function – обратная функция
        \item composition of functions – композиция функций
        \item domain of a function – область определения функции
        \item codomain of a function – область значений функции
    \end{itemize}
    
    \item \textbf{PHRASES}
    \begin{itemize}
        \item the function maps A to B – функция отображает A в B
        \item the function is one-one – функция инъективна
        \item the function is onto – функция сюръективна
        \item the function is bijective – функция биективна
        \item the function is invertible – функция обратима
    \end{itemize}
\end{itemize}

\subsection{image – образ}

\begin{itemize}
    \item \textbf{ADJ.}
    \begin{itemize}
        \item function image – образ функции
        \item image set – множество образов
    \end{itemize}
    
    \item \textbf{IMAGE + NOUN}
    \begin{itemize}
        \item image of a set – образ множества
        \item image of a point – образ точки
        \item image of a transformation – образ преобразования
    \end{itemize}
    
    \item \textbf{PHRASES}
    \begin{itemize}
        \item the image under a function – образ при отображении функции
        \item the image of the domain – образ области определения
        \item the image of a transformation – образ преобразования
    \end{itemize}
\end{itemize}

\subsection{inverse – обратный}

\begin{itemize}
    \item \textbf{ADJ.}
    \begin{itemize}
        \item inverse function – обратная функция
        \item inverse mapping – обратное отображение
        \item inverse transformation – обратное преобразование
    \end{itemize}
    
    \item \textbf{INVERSE + NOUN}
    \begin{itemize}
        \item inverse of a function – обратная функция
        \item inverse of a transformation – обратное преобразование
    \end{itemize}
    
    \item \textbf{PHRASES}
    \begin{itemize}
        \item the inverse of a function – обратная функция
        \item the inverse of a transformation – обратное преобразование
        \item the function has an inverse – функция имеет обратную
    \end{itemize}
\end{itemize}

\subsection{composition – композиция}

\begin{itemize}
    \item \textbf{ADJ.}
    \begin{itemize}
        \item function composition – композиция функций
        \item transformation composition – композиция преобразований
    \end{itemize}
    
    \item \textbf{COMPOSITION + NOUN}
    \begin{itemize}
        \item composition of functions – композиция функций
        \item composition of transformations – композиция преобразований
    \end{itemize}
    
    \item \textbf{PHRASES}
    \begin{itemize}
        \item the composition of two functions – композиция двух функций
        \item the composition of transformations – композиция преобразований
        \item the composition is associative – композиция ассоциативна
    \end{itemize}
\end{itemize}

\subsection{transformation – преобразование}

\begin{itemize}
    \item \textbf{ADJ.}
    \begin{itemize}
        \item linear transformation – линейное преобразование
        \item geometric transformation – геометрическое преобразование
        \item translation transformation – преобразование сдвига
        \item reflection transformation – преобразование отражения
        \item rotation transformation – преобразование поворота
    \end{itemize}
    
    \item \textbf{TRANSFORMATION + NOUN}
    \begin{itemize}
        \item transformation of the plane – преобразование плоскости
        \item transformation rule – правило преобразования
        \item transformation matrix – матрица преобразования
    \end{itemize}
    
    \item \textbf{PHRASES}
    \begin{itemize}
        \item the transformation maps A to B – преобразование отображает A в B
        \item the transformation is a rotation – преобразование является поворотом
        \item the transformation is a reflection – преобразование является отражением
        \item the transformation is a translation – преобразование является сдвигом
    \end{itemize}
\end{itemize}

\subsection{domain – область определения}

\begin{itemize}
    \item \textbf{ADJ.}
    \begin{itemize}
        \item function domain – область определения функции
        \item domain of definition – область определения
    \end{itemize}
    
    \item \textbf{DOMAIN + NOUN}
    \begin{itemize}
        \item domain of a function – область определения функции
        \item domain of a transformation – область определения преобразования
    \end{itemize}
    
    \item \textbf{PHRASES}
    \begin{itemize}
        \item the domain of a function – область определения функции
        \item the domain of real numbers – область действительных чисел
        \item the domain is restricted – область определения ограничена
    \end{itemize}
\end{itemize}

\subsection{codomain – область значений}

\begin{itemize}
    \item \textbf{ADJ.}
    \begin{itemize}
        \item function codomain – область значений функции
    \end{itemize}
    
    \item \textbf{CODOMAIN + NOUN}
    \begin{itemize}
        \item codomain of a function – область значений функции
    \end{itemize}
    
    \item \textbf{PHRASES}
    \begin{itemize}
        \item the codomain of a function – область значений функции
        \item the codomain of real numbers – область действительных чисел
        \item the codomain coincides with the image – область значений совпадает с образом
    \end{itemize}
\end{itemize}

\subsection{identity function – тождественная функция}

\begin{itemize}
    \item \textbf{ADJ.}
    \begin{itemize}
        \item identity function on a set – тождественная функция на множестве
    \end{itemize}
    
    \item \textbf{IDENTITY FUNCTION + NOUN}
    \begin{itemize}
        \item identity function rule – правило тождественной функции
    \end{itemize}
    
    \item \textbf{PHRASES}
    \begin{itemize}
        \item the identity function maps each element to itself – тождественная функция отображает каждый элемент в себя
        \item the identity function is bijective – тождественная функция биективна
    \end{itemize}
\end{itemize}

\subsection{parametrisation – параметризация}

\begin{itemize}
    \item \textbf{ADJ.}
    \begin{itemize}
        \item curve parametrisation – параметризация кривой
        \item unit circle parametrisation – параметризация единичной окружности
    \end{itemize}
    
    \item \textbf{PARAMETRISATION + NOUN}
    \begin{itemize}
        \item parametrisation of a curve – параметризация кривой
        \item parametrisation of a surface – параметризация поверхности
    \end{itemize}
    
    \item \textbf{PHRASES}
    \begin{itemize}
        \item the parametrisation of the unit circle – параметризация единичной окружности
        \item the parametrisation is smooth – параметризация гладкая
    \end{itemize}
\end{itemize}

\subsection{distance function – функция расстояния}

\begin{itemize}
    \item \textbf{ADJ.}
    \begin{itemize}
        \item distance function in the plane – функция расстояния на плоскости
        \item distance function from the origin – функция расстояния от начала координат
    \end{itemize}
    
    \item \textbf{DISTANCE FUNCTION + NOUN}
    \begin{itemize}
        \item distance function rule – правило функции расстояния
    \end{itemize}
    
    \item \textbf{PHRASES}
    \begin{itemize}
        \item the distance function from the origin – функция расстояния от начала координат
        \item the distance function is continuous – функция расстояния непрерывна
    \end{itemize}
\end{itemize}

\subsection{restriction – ограничение}

\begin{itemize}
    \item \textbf{ADJ.}
    \begin{itemize}
        \item function restriction – ограничение функции
        \item domain restriction – ограничение области определения
    \end{itemize}
    
    \item \textbf{RESTRICTION + NOUN}
    \begin{itemize}
        \item restriction of a function – ограничение функции
        \item restriction to a subset – ограничение на подмножество
    \end{itemize}
    
    \item \textbf{PHRASES}
    \begin{itemize}
        \item the restriction of a function to a subset – ограничение функции на подмножество
        \item the restriction is one-one – ограничение инъективно
    \end{itemize}
\end{itemize}

\subsection{bijection – биекция}

\begin{itemize}
    \item \textbf{ADJ.}
    \begin{itemize}
        \item bijective function – биективная функция
    \end{itemize}
    
    \item \textbf{BIJECTION + NOUN}
    \begin{itemize}
        \item bijection between sets – биекция между множествами
    \end{itemize}
    
    \item \textbf{PHRASES}
    \begin{itemize}
        \item the function is a bijection – функция является биекцией
        \item the bijection is invertible – биекция обратима
    \end{itemize}
\end{itemize}

\subsection{injective – инъективный}

\begin{itemize}
    \item \textbf{ADJ.}
    \begin{itemize}
        \item injective function – инъективная функция
    \end{itemize}
    
    \item \textbf{PHRASES}
    \begin{itemize}
        \item the function is injective – функция инъективна
        \item the mapping is injective – отображение инъективно
    \end{itemize}
\end{itemize}

\subsection{surjective – сюръективный}

\begin{itemize}
    \item \textbf{ADJ.}
    \begin{itemize}
        \item surjective function – сюръективная функция
    \end{itemize}
    
    \item \textbf{PHRASES}
    \begin{itemize}
        \item the function is surjective – функция сюръективна
        \item the mapping is surjective – отображение сюръективно
    \end{itemize}
\end{itemize}


\section{Section}

\subsection{proof – доказательство}

\begin{itemize}
    \item \textbf{ADJ.}
    \begin{itemize}
        \item direct proof – прямое доказательство
        \item indirect proof – косвенное доказательство
        \item rigorous proof – строгое доказательство
        \item constructive proof – конструктивное доказательство
        \item existence proof – доказательство существования
    \end{itemize}
    
    \item \textbf{PROOF + NOUN}
    \begin{itemize}
        \item proof technique – метод доказательства
        \item proof structure – структура доказательства
        \item proof strategy – стратегия доказательства
        \item proof validity – корректность доказательства
    \end{itemize}
    
    \item \textbf{NOUN + PROOF}
    \begin{itemize}
        \item induction proof – доказательство по индукции
        \item contradiction proof – доказательство от противного
        \item existence proof – доказательство существования
        \item uniqueness proof – доказательство единственности
    \end{itemize}
    
    \item \textbf{PHRASES}
    \begin{itemize}
        \item the proof of the theorem – доказательство теоремы
        \item to construct a proof – построить доказательство
        \item to complete the proof – завершить доказательство
        \item proof by exhaustion – доказательство перебором
    \end{itemize}
\end{itemize}

\subsection{theorem – теорема}

\begin{itemize}
    \item \textbf{ADJ.}
    \begin{itemize}
        \item fundamental theorem – фундаментальная теорема
        \item general theorem – общая теорема
        \item special theorem – частная теорема
        \item converse theorem – обратная теорема
        \item uniqueness theorem – теорема единственности
    \end{itemize}
    
    \item \textbf{THEOREM + NOUN}
    \begin{itemize}
        \item theorem statement – формулировка теоремы
        \item theorem proof – доказательство теоремы
        \item theorem application – применение теоремы
        \item theorem generalization – обобщение теоремы
    \end{itemize}
    
    \item \textbf{PHRASES}
    \begin{itemize}
        \item to state a theorem – сформулировать теорему
        \item to prove a theorem – доказать теорему
        \item to apply a theorem – применить теорему
        \item under the conditions of the theorem – в условиях теоремы
    \end{itemize}
\end{itemize}

\subsection{statement – утверждение}

\begin{itemize}
    \item \textbf{ADJ.}
    \begin{itemize}
        \item mathematical statement – математическое утверждение
        \item true statement – истинное утверждение
        \item false statement – ложное утверждение
        \item variable statement – утверждение с переменными
    \end{itemize}
    
    \item \textbf{STATEMENT + NOUN}
    \begin{itemize}
        \item statement form – форма утверждения
        \item statement validity – истинность утверждения
    \end{itemize}
    
    \item \textbf{PHRASES}
    \begin{itemize}
        \item to make a statement – сделать утверждение
        \item the statement holds – утверждение верно
        \item the statement fails – утверждение неверно
    \end{itemize}
\end{itemize}

\subsection{implication – импликация}

\begin{itemize}
    \item \textbf{ADJ.}
    \begin{itemize}
        \item logical implication – логическая импликация
        \item true implication – истинная импликация
        \item false implication – ложная импликация
    \end{itemize}
    
    \item \textbf{IMPLICATION + NOUN}
    \begin{itemize}
        \item implication hypothesis – гипотеза импликации
        \item implication conclusion – заключение импликации
    \end{itemize}
    
    \item \textbf{PHRASES}
    \begin{itemize}
        \item if P then Q – если P, то Q
        \item P implies Q – P влечёт Q
        \item P only if Q – P только если Q
    \end{itemize}
\end{itemize}

\subsection{equivalence – эквивалентность}

\begin{itemize}
    \item \textbf{ADJ.}
    \begin{itemize}
        \item logical equivalence – логическая эквивалентность
        \item true equivalence – истинная эквивалентность
    \end{itemize}
    
    \item \textbf{EQUIVALENCE + NOUN}
    \begin{itemize}
        \item equivalence relation – отношение эквивалентности
    \end{itemize}
    
    \item \textbf{PHRASES}
    \begin{itemize}
        \item P if and only if Q – P тогда и только тогда, когда Q
        \item P is equivalent to Q – P эквивалентно Q
    \end{itemize}
\end{itemize}

\subsection{negation – отрицание}

\begin{itemize}
    \item \textbf{ADJ.}
    \begin{itemize}
        \item logical negation – логическое отрицание
    \end{itemize}
    
    \item \textbf{NEGATION + NOUN}
    \begin{itemize}
        \item negation of a statement – отрицание утверждения
    \end{itemize}
    
    \item \textbf{PHRASES}
    \begin{itemize}
        \item it is not the case that P – неверно, что P
        \item to negate a statement – отрицать утверждение
    \end{itemize}
\end{itemize}

\subsection{contrapositive – контрапозиция}

\begin{itemize}
    \item \textbf{ADJ.}
    \begin{itemize}
        \item logical contrapositive – логическая контрапозиция
    \end{itemize}
    
    \item \textbf{PHRASES}
    \begin{itemize}
        \item the contrapositive of "if P then Q" is "if not Q then not P"
        \item контрапозиция "если P, то Q" есть "если не Q, то не P"
    \end{itemize}
\end{itemize}

\subsection{counterexample – контрпример}

\begin{itemize}
    \item \textbf{ADJ.}
    \begin{itemize}
        \item valid counterexample – корректный контрпример
    \end{itemize}
    
    \item \textbf{PHRASES}
    \begin{itemize}
        \item to provide a counterexample – привести контрпример
        \item to disprove by counterexample – опровергнуть контрпримером
    \end{itemize}
\end{itemize}

\subsection{quantifier – квантор}

\begin{itemize}
    \item \textbf{ADJ.}
    \begin{itemize}
        \item universal quantifier – квантор всеобщности ($\forall$)
        \item existential quantifier – квантор существования ($\exists$)
    \end{itemize}
    
    \item \textbf{PHRASES}
    \begin{itemize}
        \item for all x, $P(x)$ – для всех x верно $P(x)$
        \item there exists x such that $P(x)$ – существует x такое, что $P(x)$
    \end{itemize}
\end{itemize}

\subsection{lemma – лемма}

\begin{itemize}
    \item \textbf{ADJ.}
    \begin{itemize}
        \item technical lemma – техническая лемма
        \item auxiliary lemma – вспомогательная лемма
    \end{itemize}
    
    \item \textbf{PHRASES}
    \begin{itemize}
        \item to prove a lemma – доказать лемму
        \item to use a lemma – использовать лемму
    \end{itemize}
\end{itemize}

\subsection{corollary – следствие}

\begin{itemize}
    \item \textbf{ADJ.}
    \begin{itemize}
        \item immediate corollary – непосредственное следствие
    \end{itemize}
    
    \item \textbf{PHRASES}
    \begin{itemize}
        \item as a corollary – в качестве следствия
        \item to derive a corollary – вывести следствие
    \end{itemize}
\end{itemize}

\subsection{conjecture – гипотеза}

\begin{itemize}
    \item \textbf{ADJ.}
    \begin{itemize}
        \item unproven conjecture – недоказанная гипотеза
        \item famous conjecture – известная гипотеза
    \end{itemize}
    
    \item \textbf{PHRASES}
    \begin{itemize}
        \item Goldbach's conjecture – гипотеза Гольдбаха
        \item to formulate a conjecture – сформулировать гипотезу
    \end{itemize}
\end{itemize}

\subsection{contradiction – противоречие}

\begin{itemize}
    \item \textbf{ADJ.}
    \begin{itemize}
        \item logical contradiction – логическое противоречие
    \end{itemize}
    
    \item \textbf{PHRASES}
    \begin{itemize}
        \item to reach a contradiction – прийти к противоречию
        \item proof by contradiction – доказательство от противного
    \end{itemize}
\end{itemize}

\section{Section}

\subsection{identity – тождество}

\begin{itemize}
    \item \textbf{ADJ.}
    \begin{itemize}
        \item mathematical identity – математическое тождество
        \item algebraic identity – алгебраическое тождество
        \item trigonometric identity – тригонометрическое тождество
    \end{itemize}
    
    \item \textbf{PHRASES}
    \begin{itemize}
        \item the identity holds for all values – тождество выполняется для всех значений
        \item to prove an identity – доказать тождество
        \item fundamental identity – фундаментальное тождество
    \end{itemize}
\end{itemize}

\subsection{binomial – бином}

\begin{itemize}
    \item \textbf{ADJ.}
    \begin{itemize}
        \item binomial coefficient – биномиальный коэффициент
        \item binomial expansion – биномиальное разложение
        \item binomial expression – биномиальное выражение
    \end{itemize}
    
    \item \textbf{PHRASES}
    \begin{itemize}
        \item binomial theorem – биномиальная теорема
        \item binomial formula – биномиальная формула
        \item binomial probability – биномиальная вероятность
    \end{itemize}
\end{itemize}

\subsection{coefficient – коэффициент}

\begin{itemize}
    \item \textbf{ADJ.}
    \begin{itemize}
        \item binomial coefficient – биномиальный коэффициент
        \item leading coefficient – старший коэффициент
        \item constant coefficient – постоянный коэффициент
    \end{itemize}
    
    \item \textbf{PHRASES}
    \begin{itemize}
        \item coefficient of expansion – коэффициент разложения
        \item to calculate coefficients – вычислять коэффициенты
        \item coefficient matrix – матрица коэффициентов
    \end{itemize}
\end{itemize}

\subsection{expansion – разложение}

\begin{itemize}
    \item \textbf{ADJ.}
    \begin{itemize}
        \item binomial expansion – биномиальное разложение
        \item Taylor expansion – разложение Тейлора
        \item series expansion – разложение в ряд
    \end{itemize}
    
    \item \textbf{PHRASES}
    \begin{itemize}
        \item expansion of powers – разложение степеней
        \item expansion formula – формула разложения
        \item to perform expansion – выполнять разложение
    \end{itemize}
\end{itemize}

\subsection{theorem – теорема}

\begin{itemize}
    \item \textbf{ADJ.}
    \begin{itemize}
        \item binomial theorem – биномиальная теорема
        \item fundamental theorem – фундаментальная теорема
        \item general theorem – общая теорема
    \end{itemize}
    
    \item \textbf{PHRASES}
    \begin{itemize}
        \item to state a theorem – сформулировать теорему
        \item to apply a theorem – применить теорему
        \item proof of theorem – доказательство теоремы
    \end{itemize}
\end{itemize}

\subsection{series – ряд}

\begin{itemize}
    \item \textbf{ADJ.}
    \begin{itemize}
        \item geometric series – геометрический ряд
        \item infinite series – бесконечный ряд
        \item convergent series – сходящийся ряд
    \end{itemize}
    
    \item \textbf{PHRASES}
    \begin{itemize}
        \item sum of series – сумма ряда
        \item series expansion – разложение в ряд
        \item to evaluate series – вычислять ряд
    \end{itemize}
\end{itemize}

\subsection{factorisation – факторизация}

\begin{itemize}
    \item \textbf{ADJ.}
    \begin{itemize}
        \item polynomial factorisation – факторизация многочлена
        \item prime factorisation – разложение на простые множители
        \item complete factorisation – полная факторизация
    \end{itemize}
    
    \item \textbf{PHRASES}
    \begin{itemize}
        \item factorisation theorem – теорема о факторизации
        \item to perform factorisation – выполнять факторизацию
        \item unique factorisation – единственность факторизации
    \end{itemize}
\end{itemize}

\subsection{polynomial – многочлен}

\begin{itemize}
    \item \textbf{ADJ.}
    \begin{itemize}
        \item cubic polynomial – кубический многочлен
        \item quadratic polynomial – квадратный многочлен
        \item irreducible polynomial – неприводимый многочлен
    \end{itemize}
    
    \item \textbf{PHRASES}
    \begin{itemize}
        \item polynomial equation – уравнение многочлена
        \item degree of polynomial – степень многочлена
        \item roots of polynomial – корни многочлена
    \end{itemize}
\end{itemize}

\subsection{root – корень}

\begin{itemize}
    \item \textbf{ADJ.}
    \begin{itemize}
        \item real root – действительный корень
        \item multiple root – кратный корень
        \item distinct roots – различные корни
    \end{itemize}
    
    \item \textbf{PHRASES}
    \begin{itemize}
        \item root of equation – корень уравнения
        \item to find roots – находить корни
        \item root system – система корней
    \end{itemize}
\end{itemize}

\subsection{degree – степень}

\begin{itemize}
    \item \textbf{ADJ.}
    \begin{itemize}
        \item polynomial degree – степень многочлена
        \item highest degree – наивысшая степень
        \item zero degree – нулевая степень
    \end{itemize}
    
    \item \textbf{PHRASES}
    \begin{itemize}
        \item degree of polynomial – степень многочлена
        \item to raise to degree – возводить в степень
        \item equation of degree – уравнение степени
    \end{itemize}
\end{itemize}

\subsection{sum – сумма}

\begin{itemize}
    \item \textbf{ADJ.}
    \begin{itemize}
        \item finite sum – конечная сумма
        \item partial sum – частичная сумма
        \item geometric sum – геометрическая сумма
    \end{itemize}
    
    \item \textbf{PHRASES}
    \begin{itemize}
        \item sum of series – сумма ряда
        \item to calculate sum – вычислять сумму
        \item sum formula – формула суммы
    \end{itemize}
\end{itemize}

\subsection{product – произведение}

\begin{itemize}
    \item \textbf{ADJ.}
    \begin{itemize}
        \item finite product – конечное произведение
        \item scalar product – скалярное произведение
        \item direct product – прямое произведение
    \end{itemize}
    
    \item \textbf{PHRASES}
    \begin{itemize}
        \item product of roots – произведение корней
        \item to compute product – вычислять произведение
        \item product notation – обозначение произведения
    \end{itemize}
\end{itemize}

\subsection{corollary – следствие}

\begin{itemize}
    \item \textbf{ADJ.}
    \begin{itemize}
        \item immediate corollary – непосредственное следствие
        \item important corollary – важное следствие
        \item useful corollary – полезное следствие
    \end{itemize}
    
    \item \textbf{PHRASES}
    \begin{itemize}
        \item as a corollary – в качестве следствия
        \item to derive corollary – выводить следствие
        \item corollary to theorem – следствие из теоремы
    \end{itemize}
\end{itemize}

\end{document}






\end{document}